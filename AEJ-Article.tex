% AEJ-Article.tex for AEA last revised 22 June 2011
\documentclass[AEJ]{AEA}
\usepackage{graphicx}
% The mathtime package uses a Times font instead of Computer Modern.
% Uncomment the line below if you wish to use the mathtime package:
%\usepackage[cmbold]{mathtime}
% Note that miktex, by default, configures the mathtime package to use commercial fonts
% which you may not have. If you would like to use mathtime but you are seeing error
% messages about missing fonts (mtex.pfb, mtsy.pfb, or rmtmi.pfb) then please see
% the technical support document at http://www.aeaweb.org/templates/technical_support.pdf
% for instructions on fixing this problem.

% Note: you may use either harvard or natbib (but not both) to provide a wider
% variety of citation commands than latex supports natively. See below.

% Uncomment the next line to use the natbib package with bibtex 
%\usepackage{natbib}

% Uncomment the next line to use the harvard package with bibtex
%\usepackage[abbr]{harvard}

% This command determines the leading (vertical space between lines) in draft mode
% with 1.5 corresponding to "double" spacing.
\draftSpacing{1.5}

\begin{document}

\title{Category Switch Analysis of French Companies  the SIRENE database}
\shortTitle{Category Switch Analysis}
\author{Pierre PILI and Alexandre GAVAUDAN\thanks{%
PILI : pierre.pili@etu.minesparis.psl.eu GAVAUDAN: alexandre.gavaudan.minesparis.psl.eu}}
\date{\today}
\pubMonth{November}
\pubYear{2022}
\JEL{}
\Keywords{}

\begin{abstract}
%INTRODUCTION
The French National Institute of Statistics and Economical Studies divides all companies in three categories,
small companies, intermediate companies and large firms.
%OBJECTIF
This study focuses on the factors driving the switches of a company from a category to another.
%METHOD
To do so, we used the SIRENE database to which we added the INPI and Infogreffe databases.
Those databases are built on the legal structure of a company which is a collection of legal units.
Therefore, we neither have information about companies nor about how legal units are related.
We built another database linking legal units running other legal units through which we analyzed the company switches.
%RESULTS
The results of our study show that the switches are hardly related to the information about legal units and seem to be a result
of financial operations that are not related to the economic reality of a company.
%CONCLUSION
Legal units is not a relevant structure to analyze companies, to make relevant studies, efforts must be put into
profiling companies and precisely linking legal units together to build objects that reflect the economic reality of each firm. 
\end{abstract}
\maketitle
\section{Contexte de l'étude}
We all have a basic intuition of what a company is, but does it corresponds to
either an economic or legal reality ? According to the INSEE, a company is
the smallest combination of legal units forming an organizational entity, 
which produces goods and sevices, and enjoys a certain amount of decisional autonomy.
It is crucial to understand that a company does not correspond to any legal reality, it is only a statistical unit.
But how does the INSEE identify companies ?
How are the statistical categories defined ?
\subsection{From legal units to companies}
A company is formed by a collection of legal units. A legal unit is defined as a
legal entity in either the private or public legislation. This legal unit can be :
\begin{itemize}
    \item A legal unit acknowledged by the law independantly from the persons or institutions
which either own it or are a member of it.
    \item A natural person who can independantly carry out an economic activity 
\end{itemize}

A legal unit must be declared to competent authorities (Clerks of the Commercial Courts, Social Security, Public Finance Directorate...) to exist.
The existence of such a legal unit depends on the choices of its owners or creators 
(for organizational, legal or fiscal reasons).
The legal unit is the legal entity registered in the SIRENE database. 
Each legal unit is associated with a unique siren number.\newline

For instance, ALBÉA is an industrial company producing packages in the cosmetic field.
As we can see in the table below, it is formed by at least 7 legal units. We do not know if other legal units are
related to this company. The company does not have to put its name in the
legal denomination of its legal units. Each unit is related to a specific economic activity.
For instance, ALBÉA TUBES FRANCE is the legal unit taking care of the factories in Europe.
\begin{table}
    \begin{tabular}{l|l}
        \hline
        siren & denominationunitelegale \\
        \hline
        310949623 & ALBEA COSMETICS FRANCE \\
        \hline
        342438785 & ALBEA SIMANDRE \\
        \hline
        377679840 & ALBEA TUBES FRANCE \\
        \hline
        400273116 & ALBEA COSMETICS SERVICES S.A.S. \\
        \hline
        648202216 & ALBEA SERVICES \\
        \hline
        842868184 & ALBEA MANAGEMENT SOLUTIONS AND SERVICES \\
        \hline
        538012881 & ALBEA EXPLOITATION \\
        \hline
    \end{tabular}
    \caption{Siren number and denomination of the legal units in the Sirene database containing the word ALBEA}
\end{table}
\subsection{Attribution d'une catégorie statistique}
The INSEE associates a certain category to each company using a statistical definition.
This definition is based on the number of employees, the turnover, and the financial statement
and is described in the appendix. The category of each legal unit is related to the company it belongs to.
It means that the INSEE must know the links between all the legal units to come up with a
realistic result. Let us have a look at the company ALBÉA. The table 2 shows that... 
\newline
\begin{table}
    \begin{tabular}{l|l|l}
        \hline
        siren & denominationunitelegale & category \\
        \hline
        310949623 & ALBEA COSMETICS FRANCE & ETI \\
        \hline
        342438785 & ALBEA SIMANDRE & ETI \\
        \hline
        377679840 & ALBEA TUBES FRANCE & ETI \\
        \hline
        400273116 & ALBEA COSMETICS SERVICES S.A.S. & ETI \\
        \hline
        648202216 & ALBEA SERVICES & ETI \\
        \hline
        842868184 & ALBEA MANAGEMENT SOLUTIONS AND SERVICES & ETI \\
        \hline
        538012881 & ALBEA EXPLOITATION & GE \\
        \hline
    \end{tabular}
    \caption{Siren number, denomination and category of the legal units in the Sirene database in 2022 containing the word ALBEA}
\end{table}
\subsection{Cadre de l'étude et hypothèses}
The idea of this study is to identify the factors driving the category switches.
The way it is defined means that it is obviously related to the size of each company,
but is this still true at the legal unit level ? Is there a way to identify companies~?
\section{Aggrégation d'une première base de données}
\subsection{Les bases de données SIRENE et Infogreffe}
\subsubsection{The SIRENE database}
SIRENE is a database built on the legal definition of economic entities
formed by legal units. Each line corresponds to a legal unit. Here is an example
of a line of the SIRENE database with a legal unit from the company ALBÉA in 2022.
\begin{table}
    \begin{tabular}{l|l|c}
        \hline
        variable & type & ALBEA SIMANDRE \\
        \hline
        siren & string & 342438785 \\
        \hline
        denominationunitelegale & string & ALBEA SIMANDRE \\
        \hline
        categoriejuridiqueunitelegale  & string & 5710 \\
        \hline
        activiteprincipaleunitelegale & string & 22.22Z \\
        \hline
        annee\_base & int & 2022 \\
        \hline
        libtrancheeffectifsunitelegale & string & 250 à 499 salariés \\
        \hline
        nombre\_etablissements & int & 1 \\
        \hline
        siret\_siege & string & 34243878500038 \\
        \hline
        plus\_ancien\_etab & string & 1997-04-30 \\
        \hline
        category & string & ETI \\
        \hline
    \end{tabular}
    \caption{An example of a line in the SIRENE database}
\end{table}

\begin{table}
    \begin{tabular}{l|c|c|c}
        \hline
        variable & relative frequency & mean & standard deviation \\
        \hline
        siren & 100\% & NA & NA \\
        \hline
        denominationunitelegale & 100\% & NA & NA \\
        \hline
        categoriejuridiqueunitelegale  \\
        \hline
        activiteprincipaleunitelegale \\
        \hline
        annee\_base & 100\% & NA & NA \\
        \hline
        libtrancheeffectifsunitelegale \\
        \hline
        nombre\_etablissements \\
        \hline
        siret\_siege \\
        \hline
        plus\_ancien\_etab \\
        \hline
        category \\
        \hline
    \end{tabular}
    \caption{Statistical analysis of the SIRENE variables}
\end{table}
\subsubsection{The INFOGREFFE database}
Infogreffe.fr is a website which hosts data about each legal units such as their turnover, their result
and their number of employees. We scrapped this website in order to have
quantitative information about legal units. The table 5 describes the variables we managed to
scrap. We then joined this table on the SIRENE database.
\begin{table}
    \begin{tabular}{l|c|c|c}
        \hline
        variable & relative frequency & mean & standard deviation \\
        \hline
        turnover & 100\% & NA & NA \\
        \hline
        result & 100\% & NA & NA \\
        \hline
        effectif & 100\% & NA & NA \\
        \hline 
    \end{tabular}
    \caption{Statistical analysis of the INFOGREFFE variables}
\end{table}
\subsubsection{Geolocalization of legal units}
In the SIRENE database we also have access to the adress of the main facility,
which we converted into a latitude and a longitude variable.
\begin{table}
    \begin{tabular}{l|c|c|c}
        \hline
        variable & relative frequency \\
        \hline
        longitude & 100\% \\
        \hline
        latitude & 100\% \\
        \hline
    \end{tabular}
    \caption{Statistical analysis of the longitude and latitude variables}
\end{table}
\subsection{Descriptive statistics}
\subsection{Modèles et résultats économétriques}
\subsection{Conclusion et limites}
The previous graphs show that despite a thorough cleaning of the database, the results remain insignificant.
The data does not seem to significantly explain the category switches.
We decided thus decided to analyze about 50 switches, to see if we could understand the factors driving them.
What we found is that we could label three types of switches : natural switches, intermediate switches and purchase based switches.
\subsubsection{Natural switches}
A natural switch occurs when a legal unit is directly associated with a company that is either
growing or shrinking enough to reach a resonable switching size according to the INSEE standards.
In our random sample, we labeled the legal unit \textit{see-d} as a natural switch. It has been created in
2014 and is a fast growing tech company. It thus seemed to us that it could be labeled as a natural switch.
This is the only natural switch we found in our sample. 
\subsubsection{Intermediate switches}
An intermediate switch occurs when the studied legal unit switches because of a natural switch of its owner.
\subsubsection{Purchase based switches}
A purchase based switch occurs when a legal unit switches because it has been bought by a company from another cateogory.
\subsubsection{Results of our qualitative approach}
\begin{table}
    \begin{tabular}{c|c|c}
        \hline
        natural & intermediate & purchase based \\
        \hline
        1\% & 90\% & 9\% \\
        \hline
    \end{tabular}
    \caption{Frequency of each label among a random sample of 50 identified switches}
\end{table}
What we see is that the switches are almost always very complex and that they correspond to
a level that goes beyond the unit legal level. It seemed to us that the legal unit level
was not a relevant approach to understand the switches.
\section{Construction d'un réseau d'unités légales}
\subsection{Méthode de construction}
We needed a way to connect the related legal units together in order to
build objects that have an economic meaning. We figured that we could
find the owner of each legal unit on the website inpi.fr.
Sometimes the owner is a person and we do not go any futher, sometimes it is a legal unit
and we can find its siren number in the SIRENE database.
Thus we associated to each legal unit its owner.
This encodes a graph which describes the links between all the legal units.
Our nodes are the legal units, a link from one node to another describes a possession relationship.
This graph is oriented, indeed, possession is not a symetrical relationship.
We finally created a new database formed by all the sub-connected graphs derived from the graph of all legal units.
Each line corresponds to a sub-connected graph we now call a structure.
We used the SIRENE variables, and added a few brought by the graph theory.
\begin{figure}
    \includegraphics[scale = 0.1]{GUSTAVIA CONSULTING.png}
    \caption{Structure owned by the company Gustavia Consulting}
\end{figure}
The previous graph has a size of 
\subsection{Descriptive statistics}
\subsection{Modèle et résultats économétriques}
\subsection{Conclusions et limites}





References here (manual or bibTeX). If you are using bibTeX, add your bib file 
name in place of BibFile in the bibliography command.
% Remove or comment out the next two lines if you are not using bibtex.
\bibliographystyle{aea}
\bibliography{BibFile}

% The appendix command is issued once, prior to all appendices, if any.
\appendix

\section{Mathematical Appendix}

\end{document}

